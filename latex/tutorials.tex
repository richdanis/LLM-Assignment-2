\section{Using Google Colab}

Here's a simple workflow to get you started with Colab:

\begin{quote}
\url{https://scribehow.com/shared/Colab_Workflow_Short__-_n5wpWVSOyizxH1i1GxOg}
\end{quote}

\noindent Colab gives you access to one GPU ``for free'', but depending on your usage and on
resource availability, the service may decide not to grant you a GPU. We thus recommend
the following:
\begin{itemize}
	\item While you're familiarizing yourself with the codebase, or thinking about how
	to solve the problems, disconnect the GPU runtime.
	\item If you do run out of GPU resources on Colab, there is a backup solution
	using the Euler cluster (see below).
\end{itemize}

\section{Notebooks, Pytorch, NumPy, GPUs}

If you have any questions or run into any issues, ask us on RocketChat.
You can also find good explanations for the things you need to know for
the lab (and a lot more) here:
\begin{itemize}[leftmargin=1.5em]
	\item \mbox{\url{https://nbviewer.org/github/cgpotts/cs224u/blob/2020-spring/tutorial_numpy.ipynb}}
	\item \mbox{\url{https://nbviewer.org/github/cgpotts/cs224u/blob/2020-spring/tutorial_pytorch.ipynb}}
	\item \mbox{\url{https://nbviewer.org/github/cgpotts/cs224u/blob/2020-spring/tutorial_jupyter_notebooks.ipynb}}
\end{itemize}

\section{Running Jupyter Notebooks on Euler}
\label{euler}

If you run out of free GPUs on Colab, there is an option to use a GPU running on the
Euler cluster as an alternative runtime.
The process is a bit more involved, and GPU resources on Euler are limited, so please only
do this if you do run out of Colab's free tier.

\subsection{Launch a Jupyter Notebook from Euler}
\begin{enumerate}
	\item Connect to Euler (see \href{https://scicomp.ethz.ch/wiki/Accessing_the_clusters}{instructions}).

	\item download the \file{start\_jupyter.sh} script:

\begin{lstlisting}[language=bash,breaklines=true]
$ curl -O https://gist.githubusercontent.com/ftramer/b519b4e3d204189b27f94b058a6c4d20/raw/start_jupyter.sh
\end{lstlisting}

	\item Start a \jupyter notebook server on an Euler node with a GPU:

\begin{lstlisting}[language=bash]
$ sbatch start_jupyter.sh
\end{lstlisting}

	This should print out the message: \mystring{Submitted batch job \{JOB\_ID\}}.

	\item You can use the command \texttt{squeue} to check if your job is running.
	You will also get an email once it starts.

	Once it is running, your job will print out information to two local files,
	\file{jupyter.out} and \file{jupyter.err}.
	The file \file{jupyter.out} should contain something like this:

\begin{lstlisting}[basicstyle=\footnotesize\ttfamily]
 Run the following command on your local machine to enable port forwarding:
 ssh -N -L 8888:{NODE_IP}:8888 {USER}@login.euler.ethz.ch
\end{lstlisting}

	\item Run the command in \file{jupyter.out} on your \textbf{local machine}.

\begin{lstlisting}[language=bash]
$ ssh -N -L 8888:{NODE_IP}:8888 {USER}@login.euler.ethz.ch
\end{lstlisting}

	\item Now open the file \file{jupyter.err} and copy the URL that is printed at
	the bottom (this may take a minute to appear).
	It should look like this:

\begin{lstlisting}
http://127.0.0.1:8888/lab?token=d70a2ae9...
\end{lstlisting}

	\item Open this URL in the browser on your local machine.
	You should now be able to see the \jupyter notebook interface.
\end{enumerate}

\subsection{Shutting down the Jupyter Notebook}

Run \code{squeue} on Euler to get the JOB\_ID of your job.
Then run: \code{scancel JOB\_ID}.
Finally, kill the SSH tunnel on your local machine by pressing \code{Ctrl+C}.